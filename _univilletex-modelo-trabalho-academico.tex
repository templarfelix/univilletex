%% univilletex modelo baseado no abntex2 por Fernando Luís da Silva <fernando@mksdev.com>
%% univille tex em http://github.com/templarfelix/univilletex/

% verso e anverso:
%\documentclass[12pt,openright,twoside,a4paper,english,french,spanish]{univille-brasil-abntex2}

% apenas verso:	
\documentclass[12pt,oneside,a4paper,english,french,spanish]{univille-brasil-abntex2} 

% ---
% PACOTES
% ---

% ---
% Pacotes fundamentais 
% ---
\usepackage{cmap}				% Mapear caracteres especiais no PDF
%\usepackage{lmodern}			% Usa a fonte Latin Modern			
\usepackage{pslatex}            % Usa a fonte TIMES NEW ROMAN
%\renewcommand{\rmdefault}{phv} % Arial
%\renewcommand{\sfdefault}{phv} % Arial

\usepackage[T1]{fontenc}		% Seleção de códigos de fonte.
\usepackage[utf8]{inputenc}		% Determina a codificação utiizada (conversão automática dos acentos)
\usepackage{makeidx}            % Cria o indice
\usepackage{hyperref}  			% Controla a formação do índice
\usepackage{lastpage}			% Usado pela Ficha catalográfica
\usepackage{indentfirst}		% Indenta o primeiro parágrafo de cada seção.
\usepackage{color}				% Controle das cores
\usepackage{graphicx}			% Inclusão de gráficos
\usepackage{pdfpages}           % Importar PDF
\usepackage{float}              % pacote float para imagens

% UNIVILLE CORRECÃO DE TITULOS

\usepackage[brazil]{babel}
\usepackage{times}
\usepackage{fixltx2e}
%\usepackage[margin=3cm,noheadfoot]{geometry}
\usepackage[left=3cm,top=3cm,right=2cm,bottom=2cm]{geometry}
\usepackage[font=normal,singlelinecheck=false]{caption}

% ---
% Pacotes de citações
% ---
\usepackage[brazilian,hyperpageref]{backref}	 % Paginas com as citações na bibl
\usepackage[alf]{abntex2cite}	% Citações padrão ABNT
\usepackage{leading} 

% IMAGEM
\usepackage{boxhandler}
\usepackage{graphicx}

% --- 
% Espaçamentos entre linhas e parágrafos 
% --- 

% O tamanho do parágrafo é dado por:
\setlength{\parindent}{1.5cm}

% Controle do espaçamento entre um parágrafo e outro:
%\setlength{\parskip}{0.3cm}  % tente também \onelineskip

% Espaçamento entre linhas de 1.5
\linespread{1.5} 

\usepackage{titlesec}


% --- 
% CONFIGURAÇÕES DE PACOTES
% --- 

% ---
% Configurações do pacote backref
% Usado sem a opção hyperpageref de backref
%\renewcommand{\backrefpagesname}{Citado na(s) página(s):~}
% Texto padrão antes do número das páginas
%\renewcommand{\backref}{}
% Define os textos da citação
%\renewcommand*{\backrefalt}[4]{
%	\ifcase #1 %
%		Nenhuma citação no texto.%
%	\or
%		Citado na página #2.%
%	\else
%		Citado #1 vezes nas páginas #2.%
%	\fi}%
% ---


% ---
% Informações de dados para CAPA e FOLHA DE ROSTO
% ---
%\titulo{Desenvolvimento de um Framework para a Asia Shipping}
\titulo{Desenvolvimento de um Framework para a 2WA}

\autor{Fernando Luís da Silva}
\local{Joinville-SC}
\data{2013}
\orientador{Prof. Msc. Paulo Marcondes Bousfield}
\coorientador{}
\instituicao{%
  Universidade da Região de Joinville - UNIVILLE
  \par
  Departamento de Informática
}
\tipotrabalho{TCE}
% O preambulo deve conter o tipo do trabalho, o objetivo, 
% o nome da instituição e a área de concentração 
\preambulo{Trabalho de Conclusão de Estágio apresentado ao curso de Sistemas de Informação da Universidade da Região de Joinville - UNIVILLE - como requisito parcial para obtenção do grau de Bacharel em Sistemas de Informação. Orientador Específico: \imprimirorientador.}
% ---


% ---
% Configurações de aparência do PDF final

% alterando o aspecto da cor azul
\definecolor{blue}{RGB}{41,5,195}

% informações do PDF
\hypersetup{
     	%pagebackref=true,
		pdftitle={\imprimirtitulo}, 
		pdfauthor={\imprimirautor},
    	pdfsubject={\imprimirpreambulo},
		pdfkeywords={PALAVRAS}{CHAVES},
	    pdfproducer={Latex}, 	% producer of the document
	    pdfcreator={\imprimirautor},
    	colorlinks=true,       		% false: boxed links; true: colored links
    	linkcolor=black,          	% color of internal links
    	citecolor=black,        		% color of links to bibliography
    	filecolor=magenta,      		% color of file links
		urlcolor=blue,
		bookmarksdepth=4
}
% ---
% compila o indice
% ---
\makeindex
% ---

% ----
% Início do documento
% ----
\begin{document}

% Retira espaço extra obsoleto entre as frases.
\frenchspacing 

% ----------------------------------------------------------
% ELEMENTOS PRÉ-TEXTUAIS
% ----------------------------------------------------------
% \pretextual

% ---
% Capa
% ---
\imprimircapa
% ---

% ---
% Folha de rosto
% (o * indica que haverá a ficha bibliográfica)
% ---
\imprimirfolhaderosto*
% ---

% ---
% Inserir a ficha bibliografica
% ---

% Isto é um exemplo de Ficha Catalográfica, ou ``Dados internacionais de
% catalogação-na-publicação''. Você pode utilizar este modelo como referência. 
% Porém, provavelmente a biblioteca da sua universidade lhe fornecerá um PDF
% com a ficha catalográfica definitiva após a defesa do trabalho. Quando estiver
% com o documento, salve-o como PDF no diretório do seu projeto e substitua todo
% o conteúdo de implementação deste arquivo pelo comando abaixo:
%
% \begin{fichacatalografica}
%     \includepdf{fig_ficha_catalografica.pdf}
% \end{fichacatalografica}
%\begin{fichacatalografica}
%	\vspace*{\fill}					% Posição vertical
%	\hrule							% Linha horizontal
%	\begin{center}					% Minipage Centralizado
%	\begin{minipage}[c]{12.5cm}		% Largura
%	
%	\imprimirautor
%	
%	\hspace{0.5cm} \imprimirtitulo  / \imprimirautor. --
%	\imprimirlocal, \imprimirdata-
%	
%	\hspace{0.5cm} \pageref{LastPage} p. : il. (algumas color.) ; 30 cm.\\
%	
%	\hspace{0.5cm} \imprimirorientadorRotulo~\imprimirorientador\\
%	
%	\hspace{0.5cm}
%	\parbox[t]{\textwidth}{\imprimirtipotrabalho~--~\imprimirinstituicao,
%	\imprimirdata.}\\
%	
%	\hspace{0.5cm}
%		1. Palavra-chave1.
%		2. Palavra-chave2.
%		I. Orientador.
%		II. Universidade xxx.
%		III. Faculdade de xxx.
%		IV. Título\\ 			
%	
%	\hspace{8.75cm} CDU 02:141:005.7\\
%	
%	\end{minipage}
%	\end{center}
%	\hrule
%\end{fichacatalografica}
% ---

% ---
% Inserir errata
% ---
%\begin{errata}
%
%Elemento opcional da \citeonline[4.2.1.2]{NBR14724:2011}. Exemplo:
%
%\vspace{\onelineskip}
%
%FERRIGNO, C. R. A. \textbf{Tratamento de neoplasias ósseas apendiculares com
%reimplantação de enxerto ósseo autólogo autoclavado associado ao plasma
%rico em plaquetas}: estudo crítico na cirurgia de preservação de membro em
%cães. 2011. 128 f. Tese (Livre-Docência) - Faculdade de Medicina Veterinária e
%Zootecnia, Universidade de São Paulo, São Paulo, 2011.

%\begin{table}[htb]
%\center
%\footnotesize
%\begin{tabular}{|p{1.4cm}|p{1cm}|p{3cm}|p{3cm}|}
% \hline
%   \textbf{Folha} & \textbf{Linha}  & \textbf{Onde se lê}  & \textbf{Leia-se}  \\
%    \hline
%    1 & 10 & auto-conclavo & autoconclavo\\
%   \hline
%\end{tabular}
%\end{table}

%\end{errata}
% ---

% ---
% Inserir folha de aprovação
% ---

% Isto é um exemplo de Folha de aprovação, elemento obrigatório da NBR
% 14724/2011 (seção 4.2.1.3). Você pode utilizar este modelo até a aprovação
% do trabalho. Após isso, substitua todo o conteúdo deste arquivo por uma
% imagem da página assinada pela banca com o comando abaixo:
%
% \includepdf{folhadeaprovacao_final.pdf}
%
\begin{folhadeaprovacao}
	
% MODELO UNIVILLE
  \begin{center}

	  \fbox{\parbox{15cm}{

	  	O aluno \imprimirautor, regularmente matriculado na 5a série do curso de Sistemas de Informação, apresentou e defendeu o presente Trabalho de Conclusão de Estágio obtendo da Banca Examinadora a média final \_\_\_\_ (\_\_\_\_\_\_\_\_\_\_\_\_\_\_\_\_\_\_) , tendo sido considerado aprovado.\\ 

		Joinville, \_\_\_\_\_\_ de \_\_\_\_\_\_\_\_\_\_\_\_\_\_\_\_\_\_\_\_\_\_ de 2013.\\ \\

		\begin{tabular}{ccc}
		\parbox{4cm}{
			\_\_\_\_\_\_\_\_\_\_\_\_\_\_\_\_\_\_\_ \\
			Prof. "A"
		} & 
		\parbox{4cm}{
			\_\_\_\_\_\_\_\_\_\_\_\_\_\_\_\_\_\_\_ \\
			Prof. "B"
		} \\\\
		\end{tabular}
	
		}}
  \end{center}
  
\end{folhadeaprovacao}
% ---

% - DO COMPROVANTE DE ESTÄGIO DA EMPRESA
%\null
%\thispagestyle{empty}%
%\newpage

% ---
% Dedicatória
% ---
% FIXME CRIAR NO PROJETO FINAL
\begin{dedicatoria}
   \vspace*{\fill}
   \centering
   \noindent
   \textit{ Este trabalho é dedicado a todos os que me apoiaram nesta jornada. } \vspace*{\fill}
\end{dedicatoria}
% ---

% ---
% Agradecimentos
% ---
\begin{agradecimentos}

	\vfill
	
	\begin{flushright}
		\parbox{8cm}{
	\textit{ Agradeço a Deus por ter me dado força e coragem para prosseguir, mesmo diante de tantos desafios. À minha namorada, família e amigos, por sempre estarem ao meu lado, me apoiando-me e incentivando-me. Aos meus colegas de turma pela troca de  experiências. Aos meus professores, mestres que me ensinaram muitos valores. E, principalmente, agradeço ao meu orientador específico Prof. NOME PROFESSOR, por todo o apoio dado durante esses anos de caminhada.
	}}
	\end{flushright}
	
\end{agradecimentos}
% ---

% ---
% Epígrafe
% ---
\begin{epigrafe}
    \vspace*{\fill}
	\begin{flushright}
		\textit{``O homem de valor nunca morre. Seus exemplos e suas obras atestam a sua imortalidade.''\\
		(Hélder Sena de Sousa)}
	\end{flushright}
\end{epigrafe}
% ---

% ---
% RESUMOS
% ---

% FIXME CRIAR NO PROJETO FINAL
% resumo em português
%
\begin{resumo}
% 
	\noindent
	RESUMO.

%
 \vspace{\onelineskip}
%    
 \noindent
 \textbf{Palavras-chave}: \textit{Framework}.
\end{resumo}

% FIXME CRIAR NO PROJETO FINAL
% resumo em inglês
\begin{resumo}[Abstract]
 \begin{otherlanguage*}{english}
	 
     \noindent
	 RESUMO EM INGLÊS.
    
	
   \vspace{\onelineskip}
 
   \noindent 
   \textbf{Keywords}: \textit{Framework}.
 \end{otherlanguage*}
\end{resumo}

% ---
% inserir lista de ilustrações
% ---
\pdfbookmark[0]{\listfigurename}{lof}
\listoffigures*
\cleardoublepage
% ---

% ---
% inserir lista de tabelas
% ---
\pdfbookmark[0]{\listtablename}{lot}
\listoftables*
\cleardoublepage
% ---

% ---
% inserir lista de abreviaturas e siglas
% ---
\begin{siglas}
    \item[Fig.] Figuras
\end{siglas}
% ---

% ---
% inserir lista de símbolos
% ---
%\begin{simbolos}
%  \item[$ \Gamma $] Letra grega Gama
%  \item[$ \Lambda $] Lambda
%  \item[$ \zeta $] Letra grega minúscula zeta
%  \item[$ \in $] Pertence
%\end{simbolos}
% ---

% ---
% inserir o sumario
% ---
\pdfbookmark[0]{\contentsname}{toc}
\tableofcontents*
\cleardoublepage
% ---

% ----------------------------------------------------------
% ELEMENTOS TEXTUAIS
% ----------------------------------------------------------
% É possível usar \textual ou \mainmatter, que é a macro padrão do memoir.  
\mainmatter

% ----------------------------------------------------------
% Introdução
% ----------------------------------------------------------
\chapter*[Introdução]{Introdução}
\addcontentsline{toc}{chapter}{Introdução}
	
	\thispagestyle{empty}
		
	Introdução

%Exemplo de Escrita no Rodapé
%\footnote{\url{http://www.latex-project.org/lppl.txt}}

% ----------------------------------------------------------
% Parte de revisãod e literatura
% ----------------------------------------------------------
%\part{Projeto}

% INCLUI CAPITULOS DA UNIVILLE

\pagestyle{univille}
% CITAÇÔES
% \citeonline[PAGINA,PAGINA]{LIVRO}
% \nocite{LIVRO}
% \cite{LIVRO}

\chapter{DEFINIÇÃO DO PROJETO}
	\thispagestyle{empty}

    \section{Identificação do estágio}

	\subsection{Caracterização do Campo de Estágio}

		\subsubsection{Dados de Identificação do Aluno}
		\medskip\hfill{\parbox{15cm}{
			\noindent Nome:  SEU NOME\newline
			Curso: Bacharelado em Sistemas de Informação\newline
			Endereço: XXX
		}

		\subsubsection{Dados de Identificação da Empresa}
		\medskip\hfill{\parbox{15cm}
		{
			\noindent Denominação: EMPRESA XXX.\newline
			Ramo de Atividade: Desenvolvimento de Sofware.\newline
			Endereço: XXXX \newline
			Fone: XXXX  e-mail: XXX@XXX.XXX
		}

		\subsubsection{Dados dos Responsáveis pelo Estágio}
		\medskip\hfill{\parbox{15cm}
		{
			\noindent Orientador de Classe:  MSc. Paulo Marcondes Bousfield \newline
			Orientador Específico:  MSc. Paulo Marcondes Bousfield \newline
			Supervisor no Campo de Estágio: SUPERVISOR EMPRESA.
		}
	
\section{Tema}

	TEMA
	
\section{Problema}

	Problema.
		
\subsection{Objetivo Geral}

	Objetivo.

\subsection{Objetivos Específicos}

	\begin{itemize}
		\item Fazer.
		\item Testar
	\end{itemize}
	
\section{Justificativa}
 
	Jusitificar.
	
\section{Metodologia}

	Descrever metodologia utilizada e ferramentas.
	

\section{Cronograma}

	Mostrar Cronograma.
	
\section{Recursos humanos, materiais e financeiros}	
	
	Mostrar tempo e gastos com o projeto.

\pagestyle{univille}
% VALIDAR PLAGIO http://plagiarisma.net
\chapter{FUNDAMENTAÇÃO TEÓRICA}
	\thispagestyle{empty}

\section{Teste}

	Teste.
	
	ex citar: \nocite{PRESSMAN} Pressmann (2011) diz: lalalalal.
	
	ex de citação + de 3 linhas:
	
	Fulano nos fala:
	\begin{citacao}
			A vida de um sistema se prolonga conforme ele vai sendo mantido e inspe-cionado. Se o sistema precisar de uma melhoria que ultrapassa o escopo da manutenção, se precisar ser substituído por uma nova versão de tecnologia ou se as necessidades de SI da organização sofrerem alterações significativas, um novo projeto terá início e o ciclo se reiniciará.
		\end{citacao}
		
\subsection{Teste 2}
	Teste 2.
	
\subsubsection{Teste 3}
	Teste 3.

	



\pagestyle{univille}
\chapter{DESCRIÇÃO PRÁTICA}
	\thispagestyle{empty}

	Descrever rapidamente o trabalho feito. 

	\section{Apresentação da Empresa}
	
		Falar da Empresa
		
\section{Desenvolvimento}
	Descrever o desenvolvimento do projeto.
		
	


% ---
% Finaliza a parte no bookmark do PDF, para que se inicie o bookmark na raiz
% ---
\bookmarksetup{startatroot}% 
% ---

% ---
% Conclusão
% ---
\chapter*[Conclusão]{Conclusão}
\addcontentsline{toc}{chapter}{Conclusão}
%

	CONCLUSÃO

%
% ----------------------------------------------------------
% ELEMENTOS PÓS-TEXTUAIS
% ----------------------------------------------------------
\postextual

% ----------------------------------------------------------
% Referências bibliográficas
% ----------------------------------------------------------
\bibliography{biblio} 	
% ----------------------------------------------------------
% Glossário
% ----------------------------------------------------------
%
% Consulte o manual da classe univilletex para orientações sobre o glossário.
%
%\glossary

% ----------------------------------------------------------
% Apêndices
% ----------------------------------------------------------

% ---
% Inicia os apêndices
% ---
%\begin{apendicesenv}
%
%% Imprime uma página indicando o início dos apêndices
%\partapendices


%\end{apendicesenv}
% ---


% ----------------------------------------------------------
% Anexos
% ----------------------------------------------------------

% ---
% Inicia os anexos
% ---
%\begin{anexosenv}

% Imprime uma página indicando o início dos anexos
%\partanexos

%\end{anexosenv}

%---------------------------------------------------------------------
% INDICE REMISSIVO
%---------------------------------------------------------------------

\printindex

\end{document}
